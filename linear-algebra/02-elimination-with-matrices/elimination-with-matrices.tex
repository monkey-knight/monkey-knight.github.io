\documentclass[green, cn, normal]{elegantnote}

\graphicspath{{./figures/}}

\title{Elimination with Matrices}
\author{\href{https://monkey-knight.github.io/}{monkey knight}}
\version{1.00}

\begin{document}
	\maketitle
	
	\section{Method of Elimination}
	消元是计算机软件解线性方程组最通用的技术。当矩阵 $A$ 可逆时,它就能找到 $A\textbf{x} = \textbf{b}$ 的解 $\textbf{x}$。课上的例子:
	\[
	A = \begin{bmatrix}
	1 & 2 & 1 \\
	3 & 8 & 1 \\
	0 & 4 & 1
	\end{bmatrix} 
	and \,
	\textbf{b}= \begin{bmatrix}
	2 \\
	12 \\
	2
	\end{bmatrix}
	\]
	
	$A$ 左上角的 $1$ 被称为第一个 $pivot$。我们拷贝第一行,然后为这一行乘上一个合适的数(在这个例子里是 $3$)而且用第二行的数减去第一行。第二行的第一个数就变成了 $0$。因此我们就将第二行第一列的数字 $3$ 消去了。
	
	下一步就是要消去第 $3$ 行第 $1$ 列的数,得到数字 $0$。在这里,它已经是 $0$ 了。
	
	第二个 $pivot$ 是现在第 $2$ 行第 $2$ 列的数 $2$。我们需要找到一个合适乘数乘以第二行来消去第 $3$ 行第 $2$ 列的数字 $4$。 第三个 $pivot$ 就是第 $3$ 行第 $3$ 列的数 $5$。
	
	我们从一个可逆的矩阵 $A$ 开始,然后得到一个上三角矩阵(upper trangular)u$U$。$U$ 的左部分全是 $0$。$pivot$ $1$,$2$,$5$ 都在 $U$ 的对角线上。
	
	\[
	A = \begin{bmatrix}
	1 & 2 & 1 \\
	3 & 8 & 1 \\
	0 & 4 & 1
	\end{bmatrix}
	\rightarrow
	\begin{bmatrix}
	1 & 2 & 1 \\
	0 & 2 & -2 \\
	0 & 4 & 1
	\end{bmatrix}
	\rightarrow
	U = 
	\begin{bmatrix}
	1 & 2 & 1 \\
	0 & 2 & -2 \\
	0 & 0 & 5
	\end{bmatrix}
	\]
	
	然后我们对向量 $\textbf{b}=\begin{bmatrix}
	2 \\ 12 \\ 2
	\end{bmatrix}$ 重复上面的乘法和减法操作。例如:我们给第一个位置的 $2$ 乘以 $3$ 然后从 $12$ 减掉乘之后的结果得到 $6$。如果我们是手动计算的话,我们可以使用 $A$ 的增广矩阵(就是将向量 $\textbf{b}$ 添加到矩阵 $A$ 中作为最后一列)。这样,消元法将等式 $A\textbf{x} = \textbf{b}$ 转换成了一个新的等式 $U\textbf{x} = \textbf{c}$。在上面的例子中,$U=\begin{bmatrix}
	1 & 2 & 1 \\
	0 & 2 & -2 \\
	0 & 0 & 5
	\end{bmatrix}$ 来自 $A$,$\textbf{c}=\begin{bmatrix}
	2 \\ 6 \\-10
	\end{bmatrix}$ 来自 $\textbf{b}$。
	
	等式 $U\textbf{x} = \textbf{c}$ 很容易通过回代法(Back Substitution) 来解答。在我们的例子中,$z = -2, y=1, x = 2$。这也正是原方程组 $A\textbf{x} = \textbf{b}$ 的解。
	
	$U$ 的行列式就是 $pivot$ 们的乘积。
	
	$pivot$ 不能是 $0$,如果在 $pivot$ 的位置是 $0$,那我们就必须要通过行与行之间进行互换来使得 $pivot$ 的位置不为 $0$。
	
	如果 $pivot$ 的位置为 $0$,而且其下面的值都是 $0$,那么矩阵 $A$ 就不是可逆的。那么这个方程组就不存在唯一解。
	
	\section{Elimination Matrices}
	
	一个
	
\end{document}